\documentclass[12pt]{book}
\usepackage[utf8]{inputenc}
\usepackage[T1]{fontenc}
\usepackage{graphicx} 
\usepackage{amsmath, amssymb} % for math features
\usepackage{hyperref} %

\title{About computational spaces}
\author{Luis Gerardo Manrique Cardona}
\date{\today}

\begin{document}
\maketitle
\tableofcontents

\chapter{Introduction}
La computacion actual se centra en el estudio de los limites de la computacion, y sobre algunas maquinas teoricas
que ayudan a comprender ciertos problemas.

Hasta ahora sin embargo, la teoria no se ha podido expandir a procesos complejos de la realidad, tales como la sociologica,
psicologia o economia, o almenos modelados de estos problemas no pertenecen al area de la computacion.

El presente trabajo introduce el concepto de espacios computacionales, el cual pretende ser un marco para expandir el espectro
de problemas donde la teoria computacional puede modelar problemas de forma eficaz, empezando por abarcar la computacion clasica 
(maquinas de turing, $\pi $ calculus y otras variantes de la maquina de turing), otras maquinas modernas autoprogramables como
las usadas en la estadistica predictiva y por ultimo como modelar problemas economicos actuales.

\chapter{Background}

Puede entenderse la teoria de espacios computacionales como una rama de la pancomputacion, teoria que predica que todo evento en
el universo, y el universo mismo puede representarse como parte de la teoria de la computacion.

En ese orden de ideas, preferiria partir el concepto de computacion en dos areas:

1. Maquinas concretas de computacion.

2. Espacios computacionales. 
\\
Donde los espacios computacionales comprenden las maquinas concretas.
\\
\\
Con esto, la teoria de maquinas de computacion se encargan de revisar modelos especificos teoricos o practicos acerca de la computacion,
mientras que los espacios computacionales demarcan una teoria transversal que especifica la forma de construir, describir y descubrir
nuevas maquinas computacionales.

De esta forma, podria entenderse los espacios computacionales como una meta-computacion.

\chapter{Methodology}

Se plantean los siguientes objetivos

\begin{enumerate}
    \item Describir las bases matematicas para los espacios computacionales
    \item Demostrar que los espacios computacionales son turing completos
    \item Mostrar algunos ejemplos de la economia y la sociologia con la nueva teoria introducida
    \item Aplicar los conceptos de los espacios computacionales a la actual computacion estadistica predictiva y demostrar aplicaciones practicas
\end{enumerate}

\chapter{Develop}
\section{Introduccion a los espacios computacionales}

Un espacio computacional sera definido como un conjunto $W$ donde pueden existir eventos $T$, y otros subespacios $S$, donde las
interacciones definidas entre estos estan definidas por: 

\begin{enumerate}
    \item $A \rightarrow B$ (el espacio $A$ puede emitir eventos a $B$, y no especifica si $B$ es modificada o no, solo establece la posibilidad)
    \item $A \rightarrow \Box B$ (el espacio $A$ puede emitir eventos a $B$, tal que $B$ sufre modificaciones permanentes)
    \item $A \rightarrow \textbackslash B$ (el espacio $A$ puede emitir eventos a $B$, tal que $B$ sufre modificaciones temporales)
    \item Sea $\frac{E}{\rightarrow}$ el conjunto de eventos que existen en $W$ y pueden ser emitidos y recibidos por todos los espacios bajo $W$ y por $W$ mismo
    \item $\frac{T}{\rightarrow} A \frac{S}{\rightarrow}$ significara que $A$ puede recibir eventos $T$ y dar como resultado eventos $S$
    \item $|W|$ sera definido como la cantidad de eventos $\frac{T}{\rightarrow}$ que puedan existir dentro de $W$, y la cantidad de duplas $A \rightarrow B$ que puedan existir.
\end{enumerate}

\section{Sobre las modificaciones que puede sufrir un espacio computacional}

Sea \(W = \{A, \frac{T}{\rightarrow}, A \frac{T_a}{\rightarrow} \Box W \}\), llamaremos $T_{a+}$ a todos los eventos de $T$ tal que $W_1, A\frac{T_{a+}}{\rightarrow}\Box W, W_2, |W_2| > |W_1|$ es decir, los eventos $T_{a+}$ causaron una nueva tupla $A \rightarrow B$ dentro de $W$ 
o añadieron elemetnos a los posibles $T$.
\\\\
Por el contrario, llamaremos $T_{a-}$ si $W_1, A\frac{T_{a-}}{\rightarrow}\Box W, W_2, |W_2| < |W_1|$ es decir, los eventos $T_{a-}$ removieron una tupla existente $A \rightarrow B$ dentro de $W$ o eliminaron elementos de los posibles $T$.
\end{document}